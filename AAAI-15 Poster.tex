\documentclass[landscape,final,a0paper,fontscale=0.285]{baposter}

\usepackage{calc}
\usepackage{graphicx}
\usepackage{amsmath}
\usepackage{amssymb}
\usepackage{relsize}
\usepackage{multirow}
\usepackage{rotating}
\usepackage{bm}
\usepackage{url}

\usepackage{graphicx}
\usepackage{multicol}

%\usepackage{times}
%\usepackage{helvet}
%\usepackage{bookman}
\usepackage{palatino}

\usepackage[english]{babel}

\newcommand{\captionfont}{\footnotesize}

\graphicspath{{images/}{../images/}}
\usetikzlibrary{calc}


\newtheorem{theorem}{Theorem}
\newtheorem{proposition}{Proposition}
\newtheorem{example}{Example}
\newtheorem{lemma}{Lemma}
\newtheorem{corollary}[theorem]{Corollary}
\newenvironment{proof}{{\bf Proof:}}{\rule{2mm}{2mm}\\[-.5em] }
\newtheorem{definition}{Definition}

\newenvironment{tight_enumerate}{
\begin{enumerate}
  \setlength{\itemsep}{0pt}%
  \setlength{\topsep}{0pt}%
  \setlength{\partopsep}{0pt}%
  \setlength{\parskip}{0pt}%
  \setlength{\parsep}{0pt}%
}{\end{enumerate}}

\newenvironment{tight_itemize}{
\begin{itemize}
  \setlength{\itemsep}{0pt}%
  \setlength{\topsep}{0pt}%
  \setlength{\partopsep}{0pt}%
  \setlength{\parskip}{0pt}%
  \setlength{\parsep}{0pt}%
}{\end{itemize}}


\newcommand{\SET}[1]  {\ensuremath{\mathcal{#1}}}
\newcommand{\MAT}[1]  {\ensuremath{\boldsymbol{#1}}}
\newcommand{\VEC}[1]  {\ensuremath{\boldsymbol{#1}}}
\newcommand{\Video}{\SET{V}}
\newcommand{\video}{\VEC{f}}
\newcommand{\track}{x}
\newcommand{\Track}{\SET T}
\newcommand{\LMs}{\SET L}
\newcommand{\lm}{l}
\newcommand{\PosE}{\SET P}
\newcommand{\posE}{\VEC p}
\newcommand{\negE}{\VEC n}
\newcommand{\NegE}{\SET N}
\newcommand{\Occluded}{\SET O}
\newcommand{\occluded}{o}

\newcommand\ie{{\it i.e.}}
\newcommand\eg{{\it e.g.}}
\newcommand\st{{\it s.t. }}
\newcommand\wrt{{\it w.r.t. }}

\newcommand{\set}[1]{\left\{#1\right\}}
\newcommand{\Not}{not \,}

\newcommand{\Atom}{Atom\!s}
\newcommand{\Lit}{Lit}

\newcommand{\LF}{L\!F}
\newcommand{\EC}{E\!C}
\newcommand{\ECC}{E\!C\!C}
\newcommand{\SL}{S\!L}
\newcommand{\DSL}{D\!S\!L}
\newcommand{\dsl}{dsl}
\newcommand{\mS}{S\!S}
\newcommand{\trn}{tr_n}
\newcommand{\trp}{tr_p}
\newcommand{\CS}{C\!S}

%%%%%%%%%%%%%%%%%%%%%%%%%%%%%%%%%%%%%%%%%%%%%%%%%%%%%%%%%%%%%%%%%%%%%%%%%%%%%%%%
%%%% Some math symbols used in the text
%%%%%%%%%%%%%%%%%%%%%%%%%%%%%%%%%%%%%%%%%%%%%%%%%%%%%%%%%%%%%%%%%%%%%%%%%%%%%%%%

%%%%%%%%%%%%%%%%%%%%%%%%%%%%%%%%%%%%%%%%%%%%%%%%%%%%%%%%%%%%%%%%%%%%%%%%%%%%%%%%
% Multicol Settings
%%%%%%%%%%%%%%%%%%%%%%%%%%%%%%%%%%%%%%%%%%%%%%%%%%%%%%%%%%%%%%%%%%%%%%%%%%%%%%%%
\setlength{\columnsep}{1.0em}
\setlength{\columnseprule}{0mm}

%%%%%%%%%%%%%%%%%%%%%%%%%%%%%%%%%%%%%%%%%%%%%%%%%%%%%%%%%%%%%%%%%%%%%%%%%%%%%%%%
% Save space in lists. Use this after the opening of the list
%%%%%%%%%%%%%%%%%%%%%%%%%%%%%%%%%%%%%%%%%%%%%%%%%%%%%%%%%%%%%%%%%%%%%%%%%%%%%%%%
\newcommand{\compresslist}{%
\setlength{\itemsep}{0pt}%
\setlength{\parskip}{0pt}%
\setlength{\parsep}{0pt}%
}

%%%%%%%%%%%%%%%%%%%%%%%%%%%%%%%%%%%%%%%%%%%%%%%%%%%%%%%%%%%%%%%%%%%%%%%%%%%%%%
%%% Begin of Document
%%%%%%%%%%%%%%%%%%%%%%%%%%%%%%%%%%%%%%%%%%%%%%%%%%%%%%%%%%%%%%%%%%%%%%%%%%%%%%

\begin{document}

%%%%%%%%%%%%%%%%%%%%%%%%%%%%%%%%%%%%%%%%%%%%%%%%%%%%%%%%%%%%%%%%%%%%%%%%%%%%%%
%%% Here starts the poster
%%%---------------------------------------------------------------------------
%%% Format it to your taste with the options
%%%%%%%%%%%%%%%%%%%%%%%%%%%%%%%%%%%%%%%%%%%%%%%%%%%%%%%%%%%%%%%%%%%%%%%%%%%%%%
% Define some colors

%\definecolor{lightblue}{cmyk}{0.83,0.24,0,0.12}
\definecolor{lightblue}{rgb}{0.145,0.6666,1}

\hyphenation{resolution occlusions}
%%
\begin{poster}%
  % Poster Options
  {
  % Show grid to help with alignment
  grid=false,
  % Column spacing
  colspacing=1em,
  % Color style
  bgColorOne=white,
  bgColorTwo=white,
  borderColor=lightblue,
  headerColorOne=black,
  headerColorTwo=lightblue,
  headerFontColor=white,
  boxColorOne=white,
  boxColorTwo=lightblue,
  % Format of textbox
  textborder=roundedleft,
  % Format of text header
  eyecatcher=true,
  headerborder=closed,
  headerheight=0.1\textheight,
%  textfont=\sc, An example of changing the text font
  headershape=roundedright,
  headershade=shadelr,
  headerfont=\Large\bf\textsc, %Sans Serif
  textfont={\setlength{\parindent}{1.5em}},
  boxshade=plain,
%  background=shade-tb,
  background=plain,
  linewidth=2pt
  }
  % Eye Catcher
  %{\includegraphics[height=5em]{images/graph_occluded.pdf}}
  {}
  % Title
  {\bf \vspace{-.2em}\textsc{Splitting a Logic Program Revisited}\vspace{0.5em}}
  % Authors
  {
  \textsc{Jianmin Ji$^a$, Hai Wan$^{b+}$, Ziwei Huo$^b$, and ZhenFeng Yuan$^b$}\\
  \vspace{.2em}
  \large
  $^a$University of Science and Technology of China, Hefei, China  \!\hspace{1em} $^b$Sun Yat-sen University, Guangzhou, China \!\hspace{1em} $^+$Email: wanhai@mail.sysu.edu.cn
  }
  % University logo
  {% The makebox allows the title to flow into the logo, this is a hack because of the L shaped logo.
    \includegraphics[height=6.5em]{images/SYSU_USTC}
  }

%%%%%%%%%%%%%%%%%%%%%%%%%%%%%%%%%%%%%%%%%%%%%%%%%%%%%%%%%%%%%%%%%%%%%%%%%%%%%%
%%% Now define the boxes that make up the poster
%%%---------------------------------------------------------------------------
%%% Each box has a name and can be placed absolutely or relatively.
%%% The only inconvenience is that you can only specify a relative position
%%% towards an already declared box. So if you have a box attached to the
%%% bottom, one to the top and a third one which should be in between, you
%%% have to specify the top and bottom boxes before you specify the middle
%%% box.
%%%%%%%%%%%%%%%%%%%%%%%%%%%%%%%%%%%%%%%%%%%%%%%%%%%%%%%%%%%%%%%%%%%%%%%%%%%%%%
    %
    % A coloured circle useful as a bullet with an adjustably strong filling
    \newcommand{\colouredcircle}{%
      \tikz{\useasboundingbox (-0.2em,-0.32em) rectangle(0.2em,0.32em); \draw[draw=black,fill=lightblue,line width=0.03em] (0,0) circle(0.18em);}}

%%%%%%%%%%%%%%%%%%%%%%%%%%%%%%%%%%%%%%%%%%%%%%%%%%%%%%%%%%%%%%%%%%%%%%%%%%%%%%
\headerbox{Problem}{name=problem,column=0,row=0}{
%%%%%%%%%%%%%%%%%%%%%%%%%%%%%%%%%%%%%%%%%%%%%%%%%%%%%%%%%%%%%%%%%%%%%%%%%%%%%%
    Lifschitz and Turner introduced the notion of the splitting set and provided a method to divide a logic program into two parts.
They showed that the task of computing the answer sets of the program can be converted into the tasks of computing the answer sets of these parts \cite{LifschitzTurner:splitting}.

    \smallskip
    \smallskip
    However, the empty set and the set of all atoms are the only two splitting sets for many programs, then these programs cannot be divided by the splitting method.

   \vspace{0.3em}
   \vspace{0.3em}
   \vspace{0.3em}
   \vspace{0.3em}
 }

%%%%%%%%%%%%%%%%%%%%%%%%%%%%%%%%%%%%%%%%%%%%%%%%%%%%%%%%%%%%%%%%%%%%%%%%%%%%%%
\headerbox{Contributions}{name=contribution, column=1, row=0, bottomaligned=problem}{
%%%%%%%%%%%%%%%%%%%%%%%%%%%%%%%%%%%%%%%%%%%%%%%%%%%%%%%%%%%%%%%%%%%%%%%%%%%%%%
   \looseness=-1
   we propose a new splitting method that allows the program to be split into two parts by an arbitrary set of atoms, while one of them, the ``top'' part, may introduce some new atoms.

   \smallskip
   \looseness=-1
   We show that the task of computing the answer sets of the program can be converted into the tasks of computing the answer sets of these parts.

   \smallskip
   \looseness=-1
   The experiments show that, the splitting result in our method is the same as the result in Lifschitz and Turner's splitting method.


   \vspace{0.3em}

   \vspace{0.3em}


  }

%%%%%%%%%%%%%%%%%%%%%%%%%%%%%%%%%%%%%%%%%%%%%%%%%%%%%%%%%%%%%%%%%%%%%%%%%%%%%%
\headerbox{Splitting Sets}{name=splittingsets,column=2,span=2,row=0,bottomaligned=problem}{
%%%%%%%%%%%%%%%%%%%%%%%%%%%%%%%%%%%%%%%%%%%%%%%%%%%%%%%%%%%%%%%%%%%%%%%%%%%%%%
    \begin{multicols}{2}
        A set $U$ of atoms is called a {\em splitting set} of a program $P$, if for each $r \in P$, $head(r) \cap U \neq \emptyset$ implies $\Atom(r) \subseteq U$.
        \par
        The {\em bottom} and {\em top} of $P$ \wrt $U$ are denoted as $b_U(P)$  and $t_U(P)$. And they are defined as :
        $b_U(P)$ = $\set{ r \in P \mid head(r) \cap U \neq \emptyset }$.
        $t_U(P)$ = $P\setminus b_U(P)$.
        \par
        And $e_U(P, X)$ is the set of rules obtained by deleting some rules from $P$ \wrt $U$ and $X$.

        \par
        A {\em solution} of $P$ \wrt $U$ is a pair $\langle X, Y\rangle$ of sets of atoms \st
        $X$ is an answer set of $b_U(P)$, and $Y$ is an answer set of $e_U(P\setminus b_U(P), X)$.

        \smallskip
        \looseness=-1
        {\bf Example 2}. Consider the logic program $P_2$:
        \par
        $ a \gets \Not d.\ \ d \gets \Not c.\ \ a \gets c, d.\ \ c \gets .$
        \par
        $\set{c, d}$ is a splitting set of $P_2$ and the bottom of $P$ \wrt $\set{c,d}$ is $b_{\set{c, d}}(P_2) = \set{ d\gets\Not c.\ \ c\gets.}$. Further more, $e_{\set{c,d}}(P_2\setminus b_{\set{c,d}}(P_2), \set{c}) = \set{ a\gets.}$. And $\langle \set{c}, \set{a}\rangle$ is a solution of $P_2$ \wrt $\set{c,d}$.\\

        \par
        {\bf Theorem 2 (Splitting Set Theorem)}. Let $U$ be a splitting set of a program $P$. A set $S$ is an answer set of $P$ iff $S = X\cup Y$ for some solution $\langle X, Y\rangle$ of $P$ \wrt $U$.
    \end{multicols}
    \vspace{0.3em}
}
%%%%%%%%%%%%%%%%%%%%%%%%%%%%%%%%%%%%%%%%%%%%%%%%%%%%%%%%%%%%%%%%%%%%%%%%%%%%%%
\headerbox{A Future Direction}{name=future,column=0,above=bottom}{
%%%%%%%%%%%%%%%%%%%%%%%%%%%%%%%%%%%%%%%%%%%%%%%%%%%%%%%%%%%%%%%%%%%%%%%%%%%%%%
   % \begin{multicols}{1}
        \smallskip
        \looseness=-1
        The idea of splitting will find many other uses, for instance in the investigation of incremental ASP solvers and forgetting.
    %\end{multicols}
    \vspace{0.3em}
  }
%%%%%%%%%%%%%%%%%%%%%%%%%%%%%%%%%%%%%%%%%%%%%%%%%%%%%%%%%%%%%%%%%%%%%%%%%%%%%%
\headerbox{References}{name=references,column=1,aligned=future,above=bottom}{
%%%%%%%%%%%%%%%%%%%%%%%%%%%%%%%%%%%%%%%%%%%%%%%%%%%%%%%%%%%%%%%%%%%%%%%%%%%%%%
    \smaller
    \bibliographystyle{ieee}
    \renewcommand{\section}[2]{\vskip 0.05em}
      \begin{thebibliography}{1}\itemsep=-0.01em
      \setlength{\baselineskip}{0.4em}
      \bibitem{LifschitzTurner:splitting}
        Lifschitz, V., and Turner, H.
        \newblock 1994.
        \newblock {Splitting a logic program.}.
        \newblock In {\em Proceedings of the 11th International Conference on Logic Programming(ICLP-94)} 23-37.
      \end{thebibliography}
    \vspace{0.3em}
  }
%%%%%%%%%%%%%%%%%%%%%%%%%%%%%%%%%%%%%%%%%%%%%%%%%%%%%%%%%%%%%%%%%%%%%%%%%%%%%%
\headerbox{Source Code}{name=sourcecode,column=2,span=2,aligned=future,above=bottom}{
%%%%%%%%%%%%%%%%%%%%%%%%%%%%%%%%%%%%%%%%%%%%%%%%%%%%%%%%%%%%%%%%%%%%%%%%%%%%%%
      \smallskip
      For more information and source code, please visit:

      \smallskip
      \url{http://ss.sysu.edu.cn/~wh/splitting.html}
     \vspace{0.3em}
  }
%%%%%%%%%%%%%%%%%%%%%%%%%%%%%%%%%%%%%%%%%%%%%%%%%%%%%%%%%%%%%%%%%%%%%%%%%%%%%%
\headerbox{Splitting with arbitrary set of atoms for NLP}{name=nlp,column=0,span=2,below=problem,above=future}{
%%%%%%%%%%%%%%%%%%%%%%%%%%%%%%%%%%%%%%%%%%%%%%%%%%%%%%%%%%%%%%%%%%%%%%%%%%%%%%
    \smallskip
    \begin{multicols}{2}
        For extending the splitting set theorem to arbitrary set of atoms for NLP, we introduce some notions. Let $U$, $X$ be sets of atoms, $P$ an NLP.
        \par
        $\EC_U(P)$ and $\ECC_U(P, X)$ are set of rules based on $\Atom(b_U(P))\setminus U$.
        \par
        The {\em in-rules} of $P$ \wrt $U$, denoted by $in_U(P)$, is the set of rules $r\in P$ \st $head(r)\cap U\neq\emptyset$ and $(body^+(r)\cup head(r)) \not\subseteq U$, the {\em out-rules}, denoted by $out_U(P)$ is similar to $in_U(P)$.

        \par
        A set of atoms $E$ is a {\em semi-loop} of $P$ \wrt $U$ if there exists a loop $L$ of $P$ \st $E= L\cap U$ and $E\subset L$.
        We denote
        \begin{multline*}
          \SL_U(P, X) = \{ E\mid \text{$E$ is a semi-loop of $P$ \wrt $U$,}\\ E\subseteq X \text{ and } R^-(E, P, X)\subseteq in_U(P)\}.
        \end{multline*}

        \par
        Finally, we propose a new {\em top} of $P$ \wrt $U$ under $X$, denoted by $t_U(P, X)$, is the union of the following sets of rules:
        \begin{tight_itemize}
          \item $P\setminus (b_U(P)\cup out_U(P))$,
          \item $\{\,x_E\gets body(r) \mid r\in in_U(P) \text{ and } r\in R^-(E,P, X)\,\}$, for each $E\in \SL_U(P, X)$,
          \item $\{\, head(r)\gets x_{E_1}, \ldots, x_{E_t}, body(r)\mid r\in out_U(P),$ for all possible $E_i\in \SL_U(P, X)$ $(1\leq i\leq t)$ \st \mbox{$body^+(r)\cap E_i\neq\emptyset\,\}$},
        \end{tight_itemize}
        where $x_E$'s are new atoms \wrt $E\in \SL_U(P, X)$.

        \looseness=-1
        A {\em solution} of $P$ \wrt $U$ is a pair $\langle X,Y\rangle$ of sets of atoms \st
        \begin{tight_itemize}
          \item $X$ is an answer set of $b_U(P)\cup \EC_U(P)$,
          \item $Y$ is an answer set of $e_U(t_U(P, X), X)\cup \ECC_U(P, X)$.
        \end{tight_itemize}


        {\bf Theorem 3 }. Let $P$ be an NLP and $U$ a set of atoms. A set $S$ is an answer set of $P$ iff $S = (X\cup Y)\cap \Atom(P)$ for some solution $\langle X, Y\rangle$ of $P$ \wrt~$U$.


        \smallskip
        \looseness=-1
        We try Hamiltonian Circuit (HC) problem in our experiment. The numbers under ``whole'' refer to the running times (in seconds) of clasp for the whole programs.
        \smallskip
        \center
        \begin{tikzpicture}[x=1\linewidth,y=-0.58\linewidth]
            \path (0,0) node{\hspace{0.1em}\includegraphics[width=.9\linewidth]{stable1}};
        \end{tikzpicture}

    \end{multicols}
    \vspace{0.0em}
  }
%%%%%%%%%%%%%%%%%%%%%%%%%%%%%%%%%%%%%%%%%%%%%%%%%%%%%%%%%%%%%%%%%%%%%%%%%%%%%%
\headerbox{DLP and Strongly Splitting}{name=dlpsset,column=2,aligned=nlp,bottomaligned=nlp}{
%%%%%%%%%%%%%%%%%%%%%%%%%%%%%%%%%%%%%%%%%%%%%%%%%%%%%%%%%%%%%%%%%%%%%%%%%%%%%%
    We extend the splitting method to DLPs. All the notions are defined the same except for
    the {\em top} of a DLP $P$ \wrt $U$ under $X$, denoted by $t_U(P, X)$. It is the union of following set of rules:
    \begin{tight_itemize}
      \item $P\setminus (b_U(P)\cup out_U(P))$,
      \item $\{\{x_E\}\cup head(r)\setminus E\gets body(r) \mid r\in in_U(P) \text{ and } r\in R^-(E,P, X)\}$, for each $E\in \SL_U(P, X)$,
      \item $\{\, head(r)\gets x_{E_1}, \ldots, x_{E_t}, body(r)\mid r\in out_U(P),$ for all possible $E_i\in \SL_U(P, X)$ $(1\leq i\leq t)$ \st $body^+(r)\cap E_i\neq\emptyset\,\}$,
    \end{tight_itemize}
%    where $x_E$'s are new atoms \wrt $E\in \SL_U(P, X)$.\\

    \par
    For any program $P'$ that does not contain atoms in $U$, the answer sets of $P\cup P'$ can be computed from the answer sets of $b_U(P)$ and a program constructed from $(P\setminus b_U(P))\cup P'$.
    However, the property does not hold for arbitrary sets of atoms in our splitting method. We introduce the {\em strong splitting method} to solve the problem.
    \looseness=-1
    {\em Strong splitting method} only replaces $SL_U(P, X)$ to $SS_U(P, X)$ with all the other notions remaining the same.
    And, we denote %$SS_U(P,X)$ which is similar to $SL_U(P,X)$ with replacing the E as nonempty subset of $U$ instead of semi-loop. 
    \begin{multline*}
    \mS_U(P, X) = \{ E \mid E \text{ is a nonempty subset of $U$},\\ E\subseteq X \text{ and } R^-(E, P, X)\subseteq in_U(P)\}.
    \end{multline*}
  }
%%%%%%%%%%%%%%%%%%%%%%%%%%%%%%%%%%%%%%%%%%%%%%%%%%%%%%%%%%%%%%%%%%%%%%%%%%%%%%
  \headerbox{Program Simplification}{name=programsimply,column=3,aligned=nlp,bottomaligned=nlp}{
%%%%%%%%%%%%%%%%%%%%%%%%%%%%%%%%%%%%%%%%%%%%%%%%%%%%%%%%%%%%%%%%%%%%%%%%%%%%%%
    Program simplification uses a consequence to simplify the program. The answer sets of the program can be computed from resulting program.
    \par
    \looseness=-1
    We define $\trp(P, L)$ to be the program obtained from $P$ by
    \par
    1. deleting each rule $r$ that has an atom $p\in head(r)$ or $p\in body^-(r)$ with $p\in L$, and
    \par
    2. replacing each rule $r$ that has an atom $p\in body^+(r)$ with $p\in L$ by the rule
      $head(r) \gets body^+(r)\setminus L, body^-(r).$
    \par
    And we define the {\em consequence top} of $P$ w.r.t $U$, denoted by $ct_U(P)$, to be the union of the following sets of rules:
    \begin{tight_itemize}
      \item $P\setminus (b_U(P)\cup out_U(P))$,
      \item $\{ \{x_E\}\cup head(r)\setminus E\gets body(r)\mid r\in in'_U(P)$ and $r\in R^-(E, P)\}$, for each $E\in \CS_U(P)$,
      \item $\{ head(r)\gets x_{E_1}, \ldots, x_{E_t}, body(r)\mid r\in out_U(P)$, for all possible $E_i\in \CS_U(P)$ $(1\leq i\leq t)$ \st $body^+(r)\cap E_i\neq\emptyset\}$,
      \item $\{ \gets\Not x_E\}$, for each $E\in \CS_U(P)$,
    \end{tight_itemize}
    
    \par
    {\bf Proposition 8 } Let set $U$ of atoms be a consequence of an NLP $P$. A set $S$ is an answer set of $P$ iff there is an answer set $S^*$ of $\trp(ct_U(P), U)$ \st $S\setminus U = S^*\cap \Atom(P)$.
    
    
   \vspace{0.3em}
  }

\end{poster}

\end{document}
